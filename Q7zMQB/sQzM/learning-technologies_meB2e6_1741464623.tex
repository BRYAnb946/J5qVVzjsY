% !TeX program = xelatex
\input{assets/packages}
\usepackage{assets/template}
\newcommand{\hidesolutions}{}

\begin{document}
{\centering به نام زیباترین}\\
	\def\ci{\perp\!\!\!\perp}

	% header{<Assignment-Number>}{<Assignment-Title>}{<Deadline-Date>}{<Gathered-by>}{<Supervised-by>}
	\header
		{۲۴ آبان ۱۴۰۳}
		{آزمون میان‌ترم اول}
		{زمان: ۳ ساعت}
		{ماهان بیهقی - پیام تائبی - امیررضا آذری}{}
	\input{assets/info}
\vspace{-3mm}
نام و نام خانوادگی: \hspace{8cm} شماره دانشجویی:
\hline
\vspace{2mm}
لطفا پاسخ هر سوال را در یک برگه‌ی جداگانه بنویسید. بالای هر برگه نام و شماره‌ دانشجویی خود را حتما قید کنید. در پایان آزمون، برگه‌ی سوال‌های مختلف را از هم جدا کنید و هر برگه را در دسته‌ی مربوط به خود قرار دهید.
هر مساله ۲۰ نمره دارد. نمره‌ی کامل آزمون ۱۰۰ است و  ۲۰ نمره اضافه دارد.

\vspace{0.5cm}
\input{questions/q1_new}
\pagebreak
\hline
نام و نام خانوادگی: \hspace{8cm} شماره دانشجویی:
\hline
\vspace{0.5cm}
\input{questions/q2_new}
\pagebreak
\hline
نام و نام خانوادگی: \hspace{8cm} شماره دانشجویی:
\hline
\vspace{0.5cm}
\input{questions/q3_new}
\pagebreak
\hline
نام و نام خانوادگی: \hspace{8cm} شماره دانشجویی:
\hline
\vspace{0.5cm}
\input{questions/q4_new}
\pagebreak
\hline
نام و نام خانوادگی: \hspace{8cm} شماره دانشجویی:
\hline
\vspace{0.5cm}
\input{questions/q5_new}
\pagebreak
\hline
نام و نام خانوادگی: \hspace{8cm} شماره دانشجویی:
\hline
\vspace{0.5cm}
\input{questions/q6_new}
\end{document}
